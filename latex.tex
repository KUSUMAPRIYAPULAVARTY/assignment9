\documentclass[journal,12pt,twocolumn]{IEEEtran}

\usepackage{setspace}
\usepackage{gensymb}

\singlespacing


\usepackage[cmex10]{amsmath}

\usepackage{amsthm}

\usepackage{mathrsfs}
\usepackage{txfonts}
\usepackage{stfloats}
\usepackage{bm}
\usepackage{cite}
\usepackage{cases}
\usepackage{subfig}

\usepackage{longtable}
\usepackage{multirow}

\usepackage{enumitem}
\usepackage{mathtools}
\usepackage{steinmetz}
\usepackage{tikz}
\usepackage{circuitikz}
\usepackage{verbatim}
\usepackage{tfrupee}
\usepackage[breaklinks=true]{hyperref}
\usepackage{graphicx}
\usepackage{tkz-euclide}

\usetikzlibrary{calc,math}
\usepackage{listings}
    \usepackage{color}                                            %%
    \usepackage{array}                                            %%
    \usepackage{longtable}                                        %%
    \usepackage{calc}                                             %%
    \usepackage{multirow}                                         %%
    \usepackage{hhline}                                           %%
    \usepackage{ifthen}                                           %%
    \usepackage{lscape}     
\usepackage{multicol}
\usepackage{chngcntr}

\DeclareMathOperator*{\Res}{Res}

\renewcommand\thesection{\arabic{section}}
\renewcommand\thesubsection{\thesection.\arabic{subsection}}
\renewcommand\thesubsubsection{\thesubsection.\arabic{subsubsection}}

\renewcommand\thesectiondis{\arabic{section}}
\renewcommand\thesubsectiondis{\thesectiondis.\arabic{subsection}}
\renewcommand\thesubsubsectiondis{\thesubsectiondis.\arabic{subsubsection}}


\hyphenation{op-tical net-works semi-conduc-tor}
\def\inputGnumericTable{}                                 %%

\lstset{
%language=C,
frame=single, 
breaklines=true,
columns=fullflexible
}
\begin{document}


\newtheorem{theorem}{Theorem}[section]
\newtheorem{problem}{Problem}
\newtheorem{proposition}{Proposition}[section]
\newtheorem{lemma}{Lemma}[section]
\newtheorem{corollary}[theorem]{Corollary}
\newtheorem{example}{Example}[section]
\newtheorem{definition}[problem]{Definition}

\newcommand{\BEQA}{\begin{eqnarray}}
\newcommand{\EEQA}{\end{eqnarray}}
\newcommand{\define}{\stackrel{\triangle}{=}}
\bibliographystyle{IEEEtran}

\providecommand{\mbf}{\mathbf}
\providecommand{\pr}[1]{\ensuremath{\Pr\left(#1\right)}}
\providecommand{\qfunc}[1]{\ensuremath{Q\left(#1\right)}}
\providecommand{\sbrak}[1]{\ensuremath{{}\left[#1\right]}}
\providecommand{\lsbrak}[1]{\ensuremath{{}\left[#1\right.}}
\providecommand{\rsbrak}[1]{\ensuremath{{}\left.#1\right]}}
\providecommand{\brak}[1]{\ensuremath{\left(#1\right)}}
\providecommand{\lbrak}[1]{\ensuremath{\left(#1\right.}}
\providecommand{\rbrak}[1]{\ensuremath{\left.#1\right)}}
\providecommand{\cbrak}[1]{\ensuremath{\left\{#1\right\}}}
\providecommand{\lcbrak}[1]{\ensuremath{\left\{#1\right.}}
\providecommand{\rcbrak}[1]{\ensuremath{\left.#1\right\}}}
\theoremstyle{remark}
\newtheorem{rem}{Remark}
\newcommand{\sgn}{\mathop{\mathrm{sgn}}}
\providecommand{\abs}[1]{\left\vert#1\right\vert}
\providecommand{\res}[1]{\Res\displaylimits_{#1}} 
\providecommand{\norm}[1]{\left\lVert#1\right\rVert}
%\providecommand{\norm}[1]{\lVert#1\rVert}
\providecommand{\mtx}[1]{\mathbf{#1}}
\providecommand{\mean}[1]{E\left[ #1 \right]}
\providecommand{\fourier}{\overset{\mathcal{F}}{ \rightleftharpoons}}
%\providecommand{\hilbert}{\overset{\mathcal{H}}{ \rightleftharpoons}}
\providecommand{\system}{\overset{\mathcal{H}}{ \longleftrightarrow}}
	%\newcommand{\solution}[2]{\textbf{Solution:}{#1}}
\newcommand{\solution}{\noindent \textbf{Solution: }}
\newcommand{\cosec}{\,\text{cosec}\,}
\providecommand{\dec}[2]{\ensuremath{\overset{#1}{\underset{#2}{\gtrless}}}}
\newcommand{\myvec}[1]{\ensuremath{\begin{pmatrix}#1\end{pmatrix}}}
\newcommand{\mydet}[1]{\ensuremath{\begin{vmatrix}#1\end{vmatrix}}}

\numberwithin{equation}{subsection}

\makeatletter
\@addtoreset{figure}{problem}
\makeatother
\let\StandardTheFigure\thefigure
\let\vec\mathbf

\renewcommand{\thefigure}{\theproblem}

\def\putbox#1#2#3{\makebox[0in][l]{\makebox[#1][l]{}\raisebox{\baselineskip}[0in][0in]{\raisebox{#2}[0in][0in]{#3}}}}
     \def\rightbox#1{\makebox[0in][r]{#1}}
     \def\centbox#1{\makebox[0in]{#1}}
     \def\topbox#1{\raisebox{-\baselineskip}[0in][0in]{#1}}
     \def\midbox#1{\raisebox{-0.5\baselineskip}[0in][0in]{#1}}
\vspace{3cm}
\title{Assignment 9}
\author{KUSUMA PRIYA\\EE20MTECH11007}

\maketitle
\newpage

\bigskip
\renewcommand{\thefigure}{\theenumi}
\renewcommand{\thetable}{\theenumi}
Download codes from 
%
\begin{lstlisting}
https://github.com/KUSUMAPRIYAPULAVARTY/assignment9
\end{lstlisting}
%
 
 \section{QUESTION}
Prove that if two homogenous systems of linear equations in two unknowns have the same solutions, then they are equivalent.
\end{align}

%

\section{Solution}
Let the two systems of homogenous equations be 
\begin{align}
 \vec{A}\vec{x}=\vec{0}\\
 \vec{B}\vec{x}=\vec{0}\\
 \implies \myvec{A_{11}&A_{12}\\A_{21}&A_{22}\\ \vdots & \vdots \\A_{n1}&A_{n2}}\myvec{x_1\\x_2}=\myvec{0\\0\\ \vdots \\0}\label{1}\\
 \text{and  }\myvec{B_{11}&B_{12}\\B_{21}&B_{22}\\ \vdots & \vdots \\B_{n1}&B_{n2}}\myvec{x_1\\x_2}=\myvec{0\\0\\ \vdots \\0}\label{2}
\end{align}
\subsection{Case 1}
Let us assume that the solution is unique.
Since they have the same solution, both $\vec{A},\vec{B}$ must have their rank as 2.\\
Let the reduced row echelon form of $\vec{A}$ be $\vec{R_1}$ and $\vec{B}$ be $\vec{R_2}$
\begin{align}
   \vec{R_1}= \myvec{1&0\\0&1\\0&0\\ \vdots & \vdots \\0&0} \\
    \vec{R_2}= \myvec{1&0\\0&1\\0&0\\ \vdots & \vdots \\0&0}
\end{align}
Consider $\vec{R_1}$\\
Performing elementary row operation $\\R_1\leftarrow R_1+R_2\times \frac{B_{12}}{B_{11}}\\$ using elementary matrix,
\begin{align}
    \vec{E_1}=\myvec{1&\frac{B_{12}}{B_{11}}&0&\hdots&0\\0&1&0\hdots&\hdots&0\\\vdots&\vdots&\vdots&\hdots&\vdots\\0&0&0&\hdots&1}\\
   \implies \vec{E_1}\vec{R_1}=\myvec{1&\frac{B_{12}}{B_{11}}\\0&1\\0&0\\ \vdots & \vdots \\0&0}
\end{align}
Performing the row operations
\begin{align}
  R_3\leftarrow R_3+R_2 \\
  R_4\leftarrow R_4+R_2\\
  \vdots\\
  R_n\leftarrow R_n+R_2
 \end{align}
 Using a product of elementary matrices
 \begin{align}
  \vec{E_2}=\myvec{1&0&0&\hdots&0\\0&1&0&\hdots&0\\0&1&1&\hdots&0\\\vdots&\vdots&\vdots&\hdots&\vdots\\0&1&0&\hdots&1}\\
 \implies \vec{E_2}(\vec{E_1}\vec{R_1})=\myvec{1&\frac{B_{12}}{B_{11}}\\0&1\\0&1\\ \vdots & \vdots \\0&1}
\end{align}
Performing the row operations
\begin{align}
  R_2\leftarrow R_2\times B_{22}-\frac{B_{21}B_{12}}{B_{11}}\\
  \vdots\\
   R_n\leftarrow R_n\times B_{n2}-\frac{B_{n1}B_{12}}{B_{11}}
\end{align}
 Using a product of elementary matrices
 \begin{align}
  \vec{E_3}=\myvec{1&0&\hdots&0\\0& B_{22}-\frac{B_{21}B_{12}}{B_{11}}&\hdots&0\\\vdots&\vdots&\hdots&\vdots\\0&0&\hdots&B_{n2}-\frac{B_{n1}B_{12}}{B_{11}}}\\
 \implies \vec{E_3}(\vec{E_2}(\vec{E_1}\vec{R_1}))=\myvec{1&\frac{B_{12}}{B_{11}}\\0&B_{22}-\frac{B_{21}B_{12}}{B_{11}}\\ \vdots & \vdots \\0&B_{n2}-\frac{B_{n1}B_{12}}{B_{11}}}
\end{align}
Performing the row operations
\begin{align}
  R_2\leftarrow R_2+B_{21} \times R_1 \\
  \vdots\\
   R_n\leftarrow R_n+B_{n1} \times R_1
\end{align}
 Using a product of elementary matrices
 \begin{align}
  \vec{E_4}=\myvec{1&0&\hdots&0\\B_{21}& 1&\hdots&0\\\vdots&\vdots&\hdots&\vdots\\B_{n1}&0&\hdots&1}\\
  \implies \vec{E_4}(\vec{E_3}(\vec{E_2}(\vec{E_1}\vec{R_1})))=\myvec{1&\frac{B_{12}}{B_{11}}\\B_{21}&B_{22}\\ \vdots & \vdots \\B_{n1}&B_{n2}}
\end{align}
Performing $R_1 \leftarrow R_1\times B_{11}$ using elementary matrix
\begin{align}
  \vec{E_5}=\myvec{B_{11}&0&\hdots&0\\0& 1&\hdots&0\\\vdots&\vdots&\hdots&\vdots\\0&0&\hdots&1}\\
  \implies \vec{E_5}(\vec{E_4}(\vec{E_3}(\vec{E_2}(\vec{E_1}\vec{R_1}))))=\myvec{B_{11}&B_{12}\\B_{21}&B_{22}\\ \vdots & \vdots \\B_{n1}&B_{n2}}=\vec{B}\\
  \implies \vec{B}=\vec{E}\vec{R_1}\\
 \text{ where  }\vec{E}=\vec{E_5}\vec{E_4}\vec{E_3}\vec{E_2}\vec{E_1}
\end{align}
is a product of elementary matrices.\\
This indicates that $\vec{B}$ is obtained by linear combinations of $\vec{R_1}$ which is a linear combination of system $\vec{A}$.Hence $\vec{B}$ is obtained through linear combinations of $\vec{A}$.

\subsection{Case 2}
Let us assume that \eqref{1},\eqref{2} have infinitely many solutions\\
So,
\begin{align}
\text{either rank}(\vec{A}) = \text{rank}(\vec{B}) = 1 \\
\text{or rank}(\vec{A})= \text{rank}(\vec{B}) = 0
\end{align}
Rank zero indicates both $\vec{A}$ and $\vec{B}$ are null matrices and are equivalent.\\
If, rank of $\vec{A}$ = rank of $\vec{B}$ = 1\\
Row reduced echelon forms of $\vec{A},\vec{B}$ become $\vec{R_1},\vec{R_2}$ 
\begin{align}
    \vec{R_1}=\myvec{1&0\\0&0\\\vdots&\vdots\\0&0}\\
    \vec{R_2}=\myvec{1&0\\0&0\\\vdots&\vdots\\0&0}
\end{align}
Hence the same approach as in case 1 yields
\begin{align}
    \vec{B}=\vec{E}\vec{R_1}
\end{align}
 where $\vec{E}$ is a product of elementary matrices.
So, \eqref{2} can be expressed as linear combinations of \eqref{1} indicating that the two systems of equations are equivalent.
\end{document}

